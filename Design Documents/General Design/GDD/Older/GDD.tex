\documentclass{scrartcl}
\usepackage[utf8]{inputenc}
\usepackage{graphicx}

% Source: https://tex.stackexchange.com/questions/227639/redefine-emph-to-be-both-bold-and-italic 
\let\emph\relax % there's no \RedeclareTextFontCommand
\DeclareTextFontCommand{\emph}{\bfseries\em}

\graphicspath{ {./img} }

\title{ijo musi\footnote{"ijo musi" can be roughly translated as "the entertaining object" or "the game thing". Please note that toki pona uses no capital letters. In this document, sentences beginning with a toki pona word - such as will often be the case when referring to the game's title - will then start with a lowercase letter.} : Game Design Document}
\subtitle{A game about learning toki pona through the exploration of mysterious ruins.}
\author{Jeremias Kuehne}

\begin{document}
	
	\maketitle
	\clearpage
	
	\tableofcontents
	\clearpage
	
	\section{Formal Aspects (REREAD)}
		\begin{itemize}
			\item Type: video game (3D).
			\item Tools used: Godot, Blender, possibly Beepbox, other software if the need arises.
			\item End product: browser game if stable enough (unlikely). Executable otherwise.
			\item Platforms: PC.
			\item Needed hardware: PC and usual devices (monitor, mouse, keyboard, sound system).
			\item Available languages: toki pona only (possibly with some \textit{very occasional} english).
			\item Genres: exploration, "knowledgevania", language learning, mystery, narrative, puzzle, walking simulator.
			\item Number of players: 1.
			\item Duration of play: a session could be as short as a few minutes or as long as it takes to complete the whole game in one sitting. The total duration of the game will depend on how many levels can be created within the time constraints of this project. Furthermore, playtests will be necessary to estimate it. There is little to no replayability.
			\item Target age group: 16+.
		\end{itemize}
	\section{Design Pillars (REREAD)}
		The game design must:
		\begin{itemize}
			\item Go about language acquisition through \emph{exposition} and not through translation\footnote{To clarify: the game should \textit{never} link a toki pona word to an english word in order to communicate its meaning. Of course, the player themself will often make that link in their head once they get the meaning of the word, but the \textit{act} of getting to that meaning should never pass through translation. For example, a fruit could be sitting near a label that reads "kili". This would lead to the player understanding that "kili" is a word that can refer to fruits. In their head, they will likely link the tooki pona word "kili" with the english word "fruit". But the game itself should not put the player in front of a text that explicitely makes that link \textit{in the game}.}.
			\item Construct the language barrier as \emph{a series of small obstacles to be overcome} and not as a list of chores to be dealt with\footnote{There are two key elements on which this distinction rests. Firstly, the difficulty: a chore is something that is merely \textit{annoying} to do, while an obstacles is something that gets one \textit{out of its comfort zone}. Secondly, the reward: chores lack rewards, whereas obstacles are what stand between the player and a desired outcome. Furthermore, chores tend to be either repetitive tasks or tasks that need to be repeated often, whereas obstacles most often need to be overcome only once or a limited number of times.}.
			\item Encourage the player to \emph{investigate} the environment to find clues and context that allow both to understand the story and to acquire toki pona as a language.
			\item Regulate access to content based on both \emph{passive} (comprehension) and \emph{active} (expression) uses of toki pona skills.
			\item Illustrate that toki pona was developped as \emph{a language for thoughts}, and not as a tool of efficient communication. 
			\item Invite the player to reflect on some questions essential to \emph{philosophy of language}\footnote{The idea here is not to educate on specific philosophical concepts or models, but rather to put the player in an environment prone to stimulate philosophical reflection.}. 
		\end{itemize}
	\section{Design Values (WIP)}
		\subsection{Target Audience and Access (CHECK IF COMPLETE)}
			ijo musi is aimed at adults with previous experience of video games and an interest for toki pona. There is no need to be both an avid player and a conlang enthusiast to appreciate the game, but at least one of these two characteristics will likely be needed to enjoy the game\footnote{A rather unfortunate but essential characteristic of ijo musi is that it is not fully playable by toki pona speakers, as they already understand the language. Hopefully, the narrative will be interesting enough to provide an untertaining experience to them, but they will inevitably miss most of the gameplay that consists in the translation and acquisition of a new language.}.\\
			As such, the main way for people to hear about the game will be posts made on various forums dedicted either to toki pona or to independant game development.\\
			The game is meant to scratch an itch similar to games like \textit{Outerl Wilds}, or \textit{Return of the Obra Dinn}, although with my current skills and resources, its design will be much less efficient and its scope will be \textit{much} smaller.
		\subsection{Conditions of Play (CHECK IF COMPLETE)}
			ijo musi is meant to be played alone in a setting that allows for immersion (without too much outside stimulation). It is meant to be played on a computer using a keyboard and a mouse. Controller support might be added if time allows it.  
		\subsection{Point of View (EVERYTHING TO DO)}
		\subsection{Challenges (CHECK IF COMPLETE)}
			The main challenges posed by the game will be the following:
			\begin{itemize}
				\item Navigating the space
				\item Deducting the meaning of toki pona words based on contextual clues
				\item Using the acquired toki pona knowledge to decipher the various texts spread across the game space
				\item Using the acquired toki pona knowledge to fill the blanks in some of the texts in order either to reveal information or to manipulate the environment
				\item Connecting the available pieces of information together to uncover the wider narrative
			\end{itemize}
		\subsection{Decision-making (CHECK IF COMPLETE)}
			The decisions made by the player will be the following:
			\begin{itemize}
				\item What thread to investigate\\
				There are no right and wrong options for this type of decisions. Even the pursuit of a thread that is not well developped in the narrative should result in the player discovering information relating to other threads.
				\item Where to go and what to look for to investigate said thread\\
				Player can choose "wrongly" relating to this type of decisions. For example: if they investigate the death of a character that was poisoned in the dining room, going to the library is unlikely to wield any useful information.
				\item When filling blanks, what word to use\\
				There are right and wrong options for this type of decisions. While each toki pona word has a wide list of possible meanings, labeling a waterfall as a "kiwen" (which can mean things such as "hard object", "metal", "stone", "solid"), will not be validated as a good answer. There will be no consequences outside of a negative feedback (such as a trembling screen).
				\item When changing words, what word to use\\
				While there are no actual wrong options relating to this type of decisions, some words will be useful or not depanding on what the player is trying to accomplish. Trying to get a door to unlock will require to use the toki pona word "open", for example.
			\end{itemize}
		\subsection{Skills, Strategies, Chance, and Uncertainty (CHECK IF COMPLETE)}
			There is no uncertainty and randomness that comes from the game side of the experience. The player might base some of their decisions on randomness, but the game itself should not react differently to the same series of inputs. Uncertainty will come from the learning experience. The player might be uncertain that they have correctly assessed the meaning of a toki pona word\footnote{In that regard, the game should contain many occasions for the player to test their hypothesis. The texts with blanks  fill this role, as they should give a positive feedback when correctly filled, and a negative feedback when incorrectly filled. To a lesser extent, the texts with changeable words also provide positive feedback, though it is less direct. In that case, the words won't be validated as "correct" or "incorrect", but they will result in an alteration of the environment that might match with the supposed meaning of the word - indirectly validating the hypothesis - or not match with the supposed meaning of the word - indirectly invalidating the hypothesis.}. They might also be uncertain that they have correctly interpreted the narrative. \\ 
			Strategies put in place by the player should relate to:
			\begin{itemize}
				\item The acquisition of toki pona
				\item The use of toki pona
				\item The navigation of the game space
				\item The investigation of narrative elements
			\end{itemize}		
		\subsection{Themes (SMALL SPOILERS) (EXPAND BASED ON JAVET)}
			CAUTION: ijo musi is a knowledge based game. Any reading made about the narrative before playing it will irremediably hinder the untertainment value of the game.
			\begin{itemize}
				\item \emph{Language as a way to relate to the world}\\
					The story goes that Sonja Lang, creator of the toki pona language, did so during a tough depressive episode as a way to try and reclaim control over her worldview. toki pona was never meant to be a tool of communication. Or if it was, this goal was always secundary to its true purpose; be a language for \textit{thoughts}.\\
					In ijo musi, there are several ways in which this aspect of toki pona comes into play:
					\begin{itemize}
						\item Mechanically:
							the player will need to learn to see the world through a toki pona lense\footnote{There are, of course, many ways to do so. It is very rare that two people would use the same toki pona sentence to describe the same scene. Nevertheless, there is a philosophy of toki pona - which allows it to function with so few words - that one must understand and adopt if one wishes to appropriately use toki pona.}. They will need, for example, to aggregate the concepts of "good" and of "simple" into the unique concept of "pona". Furthermore, the use of toki pona will allow the player to alter the environment, symbolically representing how toki pona allows us to change our representation of the world.
						\item Narratively\footnote{A basic understanding of the game's plot is required to fully understand the next paragraph. A synopsis is available in the "World, Synopsis, Characters" part of this document.}: 
							this theme is central to the narrative of ijo musi. All the characters actively try to use toki pona to shape their own way of thinking. The Enlightened One allegedly succeeded in reaching a form of enlightenment through it. The Rigid One and the Loose One both try to follow in their footsteps. Furthermore, they both build the Shaped One as an attempt to create a "pure" or "perfect" mind according to their faith; a mind that would natively think in toki pona. However, toki pona proves to be as much a constraint on the Shaped One's mind as any other native language might be on a human mind.
						\item Visually: 
							the game will use an ascii shader on top of a pixelation shader. This will result in the graphics of the game being rendered through ascii characters. The catch is that, while these character will start as letters from the latin alphabet, they will slowly change to be sitelen pona (logographs used to write toki pona). This will visually represent how the player character slowly learns to see the world "through a toki pona lense".
					\end{itemize}		
				\item Hubris\\
					All the characters in the game display a form of hubris. The Enlightened One claims to have gained access to some absolute truth and founded a whole temple dedicated to their way of thinking. The Rigid One and the Loose One both consider themselves responsible and powerful enought to be able to create a whole new form of consciousness through the Shaped One. They succeed in creating an artificial consciousness, but they fail to account for its own specificity. The Shaped One's mind eludes them, and instead of studying and trying to understand this new form of consciousness, they merely project their own desires on it. The Rigid One tries to make them an absolute incarnation of the cult doctrines, whereas the Loose One pretends to interact with them as equals, invisibilizing the power dynamics that result from them being their creator. The Shaped one also exemplifies hubris, as after they flee the temple, they themself create several artificial consciousnesses using the same magic as the cult.
				\item Independance and control
			\end{itemize}
		\subsection{Experience (REREAD)}
			The player will:
			\begin{itemize}
				\item Move through the game space. \\
				At all stages of play, they should feel:
				\begin{itemize}
					\item Curious
					\item Immersed
					\item Inquisitive
				\end{itemize}
				During early stages of play, they should feel: 
				\begin{itemize}
					\item Disoriented
					\item Intimidated
					\item Overwhelmed
					\item Wary
				\end{itemize}
				During late stages of play, they should feel:
				\begin{itemize}
					\item Confident
					\item In control
				\end{itemize}
				\item Investigate scenes to find contextual clues\footnote{These clues might give insight about the plot (such as locket containing a portrait) or help formulate theories about toki pona and its vocabulary (such as a sign displaying a message in toki pona akin to "caution: dangerous cliff ahead" in front of said cliff). As much as possible, clues should do both at once (such as a name written in toki pona on the door of a character's room).}.\\
				At all stages of play, they should feel:
				\begin{itemize}
					\item Curious
					\item Empathetic (the player does not have to identify with the characters or be "on their side", but they should care about each character's actions and the consequences of said actions)
					\item Inquisitive
					\item Smart (they should not feel like they merely stumble on knowledge or solutions to puzzles, but like they conquer them)
					\item Rewarded (by their better understanding of the plot, by a strenghtening of their toki pona skills, and by their ability to solve puzzles thanks to environmental clues)
				\end{itemize}
				During early stages of play, they should feel: 
				\begin{itemize}
					\item Lost
				\end{itemize}
				During late stages of play, they should feel:
				\begin{itemize}
					\item "In on" what happened (they should be able to link the state of the environment with the events that transpired in it)
				\end{itemize}
				Occasionnaly, they should feel:
				\begin{itemize}
					\item Intrusive (they should feel like they are snooping where they don't belong)
					\item Surprised
					\item Various emotions towards characters (mostly: sad, angry, amused, annoyed)
				\end{itemize}
				\item Read texts of various forms and types (mostly dialogs) that are dispersed throughout the game space.\\
				At all stages of play, they should feel:
				\begin{itemize}
					\item Active (they should feel like they are extracting information rather than receiving it\footnote{For a deeper analysis, see the push/pull distinction made by Kelsey Beachum in her GDC talk \textit{Sparking Curiosity-Driven Exploration Through Narrative in 'Outer Wilds'} (Beachum, 2022).})
					\item Eager to learn more about the story and characters
					\item Eager to learn toki pona
					\item Frustrated (there should always be a little piece of information that resists interpretation, up until the late stages of play)
					\item Rewarded (sometimes, information should be \textit{identifiable} as a reward for language learning, as a way to communicate to the player that their efforts are not in vain\footnote{Of course, to \textit{actually} identify the reward as such, the player should both have a basic knowledge of game design, and make an active effort to analyse the game mechanics. The point here is more about feelings rather than knowledge; the player should \textit{feel} like they are being rewarded for an expected behavior.})
					\item Rewarded (sometimes, information should \textit{function} as a reward, but the player should feel like they hacked a system to gain access to a secret they were not meant to know yet, rather than feel like they are recompensed for expected language learning efforts)
				\end{itemize}
				During early stages of play, they should feel: 
				\begin{itemize}
					\item Humbled (when facing the more complex toki pona texts for the first time, the player should experience feelings akin to the ones felt when getting obliterated by a high level ennemy in an RPG)
					\item Motivated (the challenge posed by the various texts should feel inviting rather than intimidating)
				\end{itemize}
				During late stages of play, they should feel:
				\begin{itemize}
					\item Closure
					\item Empowered (they should realize the long way they have come)
				\end{itemize}
				Occasionnaly, they should feel:
				\begin{itemize}
					\item Various emotions towards characters (mostly: sad, angry, amused, annoyed)
				\end{itemize}
				\item Fill the blanks or change the words in some of the toki pona texts in order to either 1) gain access to new texts or 2) manipulate the environment\footnote{For an explanation of these game mechanics, see the "Features" section of this document.}.\\
				The emotions felt by the player when manipulating texts are expected to be very similar to the ones felt when reading them. For emotions relating to the story and characters, it is expected that the player will generaly feel emotions of a lesser intensity when manipulating texts as opposed to reading (as the aim of the texts meant to be manipulated is often to test active skills rather than relay narrative elements). On the other hand, feelings relating to the game mechanics - such as the feeling of reward - will tend to be greater when correctly applying active toki pona skills, as the reward will be more tangible and direct than when using passive skills (in the case of a correctly filled text for example, visual and auditive rewards will be displayed, and a new dialog will become accessible, whereas no such things ocur in the case of a correctly interpreted dialog).
			\end{itemize}
	\section{Features (WIP)}
		FROM OLD FOOTNOTE: In the first case, correctly filling the text relating to an object would make a dialog appear. For example, labelling an apple as "kili" (in this context: "fruit"), would give access to the transcript of a dialog relating to said apple between two characters. In the second case, there only would be a few possible words for each changeable space, and changing them would alter the environment. For example, a locked door would initially be labeled as "lupa pini", (which can be translated as "closed door") The player could change the word "pini" (in this context: "closed") for the word "open" (in this context: "open"), resulting in the door unlocking.
		\subsection{General Overview (WIP)}
			The player character will find themself trapped in an old temple they are exploring. They will be able to move in 4 directions (forward, right, backward, left), and look around with a camera controlled with basic fps camera controls. There will be a basic collision and gravity system, but (likely) no possibility to jump. The player will have a tool at their disposal in the form of a tablet made of a screen framed by thicker edges\footnote{The nature of the tool might change during development as technical challenges might arise.}. Clicking on a button (likely the right mouse button) will lift the tablet in front of the player camera. Trhough the tablet, the camera will highlight interactive objects in a yellow, red, green, or blue outline. Clicking a button  (likely the left mouse button) while aiming at a highlighted object will cause the tablet screen to display an interface bearing a toki pona text representing the highlighted object. Doing so will freeze the camera and display the mouse cursor. The type of the interface will differ depending on the highlight color.\\
			For objects highlighted in yellow, some of the text's words will be changeable. To change a word, the user will click on it (with the left mouse button) and either select a word from a displayed list, or type-in the word they want themself. Only a restrained list of words will be available for each changeable word. Pressing enter or clicking anywhere on the display other than on the selected word will validate the word (not the whole text). It will be impossible to validate a word if it is not from the available list. Trying to do so would result in the word going back to the one it was before selection. Once the user is satisfied with the text, they can validate it by clicking a "process" button on the bottom right of the interface (alternatively, they might press enter when no word is selected to validate the whole text). Once the text is validated, the highlighted object will change to match its new toki pona description. As an example, water might freeze if it is labeled as "telo kiwen" (in this context: "solid water") or "telo lete" (in this context: "cold water").\\
			For objects highlighted in red, the interface will display a text with blank spaces. The player will be able to fill in this blanks in a manner similar to the one described for yellow-highlighted objects, with the key difference being that the list of available word is broader, as it contains all toki pona words. When the player tries to validate the text, two things can happen. If the words are wrong, the UI will shake. If the words are correct, the UI will close, a rewarding sound will play, the cursor will disappear, the camera will unfreeze, and a dialog will appear in the game space. The dialog will take the form of floating letters in space, and will be visible only through the tablet screen.\\
			Green and blue highlight will apply to objects whit correctly filled texts. They will appear green if the dialog is currently not displayed in the space, and blue if it is. The interface will show the correct, now un-modifiable text, and a button will allow to toggle the dialog visibility in space\footnote{This is not an optimal way to toggle text. As such, this system is succeptible to change during development, as I suspect playtests will reveal it to be too bothersome. As of now though it is the best comprimise I have found between a simple to toggle dialog and an easy to access and readable object text.}.
			The features are divided into four categories: "must have", "should have", "nice to have", and "bonus".
		\subsection{Must Have Features (WIP)}
			Those are the features needed for the game to run at all.
			\begin{itemize}
				\item A character that can walk and rotate in 3d space
				\item An FPS camera that can look around
				\item Collision handling for the player character, the environment, and objects
				\item A global light
				\item Low-poly 3D assests for key elements (the main tool used by the player, the walls, some furniture, some other objects)
				
			\end{itemize}
		\subsection{Should Have Features (WIP)}
			Those are the features needed for the game to be any fun
		\subsection{Nice to Have Features (WIP)}
			Those are the features that could transform an okay-ish game into a good one.
		\subsection{Bonus Features (WIP)}
			Those are features that are shiny and appealing, but not realistic to implement within the resource constraints of the project.
	\section{Gameloops (EVERYTHING TO DO)}
	\section{Graphics and Concept Art (EVERYTHING TO DO)}
	\section{UX and UI (EVERYTHING TO DO)}
	\section{World, Synopsis, Characters (HEAVY SPOILERS) (WIP)}
		CAUTION: ijo musi is a knowledge based game. Any reading made about the narrative before playing it will irremediably hinder the untertainment value of the game.
		\subsection{World (REREAD)}
			Somewhere, in an unnamed mountain range lies an old decrepyt temple. The world that lies outside the temple is of little importance, as the whole narrative is contained inside its walls.
		\subsection{Synopsis (WIP)}
		WRITE EVENTS THAT PRECEDE NARRATIVE.
		THEN ADD THE FOUR MAIN MYSTERIES HERE (skeleton, fate, rituals, identity). MAKE IT SO YOU HAVE A STARTING AND END POINT CLEARLY SHOWN IN THE GDD FOR EACH OF THE 4. HIGHLIGHT KEY SCENES.
		\subsection{Characters (WIP)}
			There are 3 main characters in the story, and a single secondary one. None of them have a definitive name at the moment, but they each are designated by a concept.
			\begin{itemize}
				\item \emph{The Enlightened One}
					is the only secondary character, as they do not take part in most dialogs. In the diegesis, they were the inventor of toki pona and the founder of the temple. 
				\item \emph{The Rigid One}
					 were the first disciple of the Enlightened One. They were rigorous and tended to follow their master's teaching to the letter.\\
					Ultimately, the conviction that they were the only one to truly understand their master's teachings lead to their fall, as they were murdered by the Shaped One. 
				\item \emph{The Loose One}
					 were the second disciple of the Enlightened One. While they still observed the proper rites dictated by the Enlightened One, they tended to have a flexible interpretation of their teachings.\\
					In the end, they were left to die by the Rigid One.
				\item \emph{The Shaped One}
					were a person created from magical clay by the Rigid One and the Loose One.\\
					In the end, they realized that neither the Rigid One nor the Loose One could truly understand their experience of life, and so they fled the temple.
			\end{itemize}
\end{document}